\documentclass[12pt] {article}
\usepackage[portuguese]{babel}
\usepackage[utf8]{inputenc}

\begin{document}

\title{Relatório Trabalho Prático C \\ $\small{Grupo 13}$}
\author{João Costa \and Leandro Salgado \and Martinho Aragão}

\maketitle

%--------------------------------
% Group Members
%--------------------------------

\begin{flushleft}
João Costa A70563 \\
FOTO
\end{flushleft}

\begin{flushleft}
Leandro Salgado A70949 \\
FOTO
\end{flushleft}

\begin{flushleft}
Martinho Aragão AXXXX \\
FOTO
\end{flushleft}

\newpage

\tableofcontents

\newpage

\section{Módulos}
Na seguinte seção aprasentam-se desenhos comentados das estruturas de dados,
todos os typedef e a documentação da API comentada, função a função.

%------------------------
%	Clients
%------------------------

\subsection{Catálogo Clientes}
\par Esta subsecção trata da API do catálogo de clientes e da sua implementação

\subsubsection{Estrutura de Dados}
\par
Para guardar os clientes lidos a partir do ficheiro resolvemos usar uma Trie pois apesar de
necessitar de mais instruções a procura é igualmente rápida

\subsubsection{Definições de dados (Typedef)}
\emph{typedef struct node * ClientsCat;} - Catálogo de Clientes

\subsubsection{Funções API}

% initClients
\noindent\textbf{ClientsCat initClients()}
\par
Inicializa a estrutura do catálogo de clientes.  Para garantir o encapsulamento de dados a função devolve um valor do tipo \emph{ClientsCat} que o utilizador da API
não sabe como está definido pois não sabe como está definida a \emph{struct node.} \\

% insertClient
\noindent\textbf{ClientsCat insertClient (ClientsCat cat, char * client)}
\par Insere um dado client, o argumento \emph{client}, na estrutura de clientes, argumento \emph{cat}.
\par Se a estrutura não tiver sido inicializada ou o cliente não existir a função retornou o valor \emph{NULL},
caso contrário devolve o valor cat, isto permite que duas variáveis trabalhem na mesma estrutura. \\

% searchClient
\noindent \textbf {Bool searchClient (char * client)}
\par Verifica se um cliente, o argumento \emph{client}, existe no catálogo de clientes.
\par A função devolve um valor do tipo Bool, definido em \emph{"Boolean.h"}, que será \emph{true} se o
cliente existir no catálogo e \emph{false} caso contrário. \\

% removeClient
\noindent \textbf{ClientsCat removeClient(ClientsCat cat, char * client)}
\par Remove um cliente, o argumento \emph{client}, de um dado catálogo de clientes, o argumento \emph{cat}.
\par A função retorna \emph{NULL} caso a estrutura não tenha sido inicializada ou o cliente seja inválido, e
retorna a própria estrutura caso contrário. \\

% searchClients
\noindent \textbf{StrList searchClients (ClientsCat cat, char init)}
\par Cria lista com todos os clientes cujo código comece por uma certa letra, o argumento \emph{init}, e que
estejam presentes no catálogo, \emph{cat}, passado como argumento.
\par Como o utilizador não conhece a definição do tipo \emph{StrList} apenas a main sabe como utilizar
o valor devolvido pela função. \\

% numOfClients
\noindent \textbf{int numOfClients (ClientsCat cat)}
\par Calcula o número de clientes presentes no catálogo, \emph{cat}, fornecido como argumento.
\par Como o utilizador não sabe como é que a estrutura de dados do catálogo está definida não consegue
calcular o número de clientes no catálogo sem recorrer a esta função. \\

% deleteCat
\noindent \textbf{ClientsCat deleteCat (ClientsCat cat)}
\par Liberta a memória ocupada pelo catálogo de clientes fornecido como argumento, \emph{cat}.
\par A função retorna NULL se a libertação de memória for bem sucedida. \\

% validateClient
\noindent \textbf{Bool validateClient (char * client)}
\par Verifica se um dado cliente, o argumento \emph{client}, é válido. Um cliente é válido se os dois primeiros
caracteres forem duas letras maiúsculas e os restantes três caracteres forem números.
\par A função retorn \emph{true} se o cliente for válido e \emph{false} se não o for.

%------------------------
%	Sales
%------------------------
\subsection{Módulos Compras}

\subsubsection{Estrutura de Dados}
\indent\par Para o Módulo de compras decidimos criar duas estruturas diferentes, apesar de isto significar um maior 
tempo  de inicialização significa também um ganho a nível de rapidez de algumas das queries.
 \par Para ambas as estruturas usamos arvóres AVL, mas numa das arvóres os nodos contêm um código de cliente 
 e um array de 12, meses, apontadores para árvores AVL em que os nodos contêm o código de cliente e a 
 quantidade comprada.
 \par Na segunda árvore os nodos contêm o código de produtos, a quantidade comprada, o número de clientes que 
 comprou o dado produto e um apontador para uma AVL com os clientes que compraram esse produto.
 
 \subsubsection{Definições de Dados (Typedef)}
 \emph{typedef struct clientNode * SalesC;} - AVL organizada por código de cliente
 \emph{typedef struct productNode * SalesP;} - AVL organizada por código de produto
 
 \subsubsection{API AVL de Clientes}
 % initSales
\noindent \textbf {SalesC InitSales ()}
\par Inicializa a AVL organizada por código de clientes, devolvendo o nodo do tipo \emph{SalesC}.
\par O utilizador apenas sabe que está a ser utilizada uma árvore AVL, não sabendo como está definido o tipo 
\emph{SalesC} nem o tipo \emph{clientNode} não tem como aceder as variáveis sem ser através das funções 
disponibilizadas pela API. \\

% insertClientSC
\noindent \textbf {SalesC insertClientSC (SalesC sales, char * client)}
\par Insere um cliente, argumento \emph{client}, na AVL, devolvendo a AVL, é necessário guardar este valor
pois a árvore pode sofrer rotações e não guardando o valor provocará erros em futuras utilizações das funções. \\

%removeClientSC
\noindent \textbf {SalesC removeClientSC (SalesC sales, char * client)}
\par Remove um cliente, argumento \emph{client}, da AVL, argumento \emph{sales}. A função retorna a nova AVL 
sem o nodo de cliente que foi especificado. \\

%insertProductSC
\noindent \textbf {SalesC insertProductSC (SalesC sales, char * client, \\ char * product, int month, int quant)}
\par Insere um produto, \emph{product}, como comprado pelo cliente, \emph{client}, num dado mês, \emph{month}, 
guardando também a quantidade comprada. Se o produto já existir a quantidade é atualizada. \\

%yearlyClients
\noindent \textbf {StrList yearlyClients (SalesC sales, StrList list)}
\par Cria uma lista com os códigos de clientes, presentes em \emph{sales}, que compraram produtos todos os 
meses do ano, guardando numa lista passada como argumento, \emph{list}. \\

%clientMonthlySales
\noindent \textbf {ProductsN clientMonthlySales (SalesC sales, char * client)}
\par Martinho faz esta sff xD. \\

%productsOnMonth
\noindent \textbf {StrList productsOnMonth (SalesC sales, char * client, int month)}
\par Cria uma lista com os produtos comprados por um dado cliente, \emph{client}, num dado mês, \emph{month}, 
caso o cliente esteja presente em \emph{sales}. \\

%topProducts
\noindent \textbf {StrList topProducts (SalesC sales, char * client)}
\par Cria uma lista com os três produtos mais comprados por um dado cliente, \emph{client}, caso esse cliente 
esteja presente em \emph{sales}. \\

%clientMonthlyPurchases
\noindent \textbf {ClientsMonth clientsMonthlyPurchases (SalesC sales)}
\par Outra para ti Martinho, sou mesmo simpático!. \\

%freeSales
\noindent \textbf {void freeSales (SalesC sales)}
\par Apaga a AVL passada como argumento \emph{sales}, libertando toda a memória usada pela mesma. \\

\end{document}