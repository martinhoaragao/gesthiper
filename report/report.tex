\documentclass[12pt] {article}
\usepackage[portuguese]{babel}
\usepackage[utf8]{inputenc}
\setlength\parindent{24pt}

\begin{document}

\title{Relatório Trabalho Prático C \\ Grupo 13}
\author{João Costa}

\maketitle

\begin{flushleft}
João Costa A70563 \newline
FOTO
\end{flushleft}

\begin{flushleft} 
Leandro Salgado A70949 \newline
FOTO
\end{flushleft}

\begin{flushleft} 
Martinho Aragão AXXXX \newline
FOTO
\end{flushleft}

\newpage

\tableofcontents

\newpage

\section{Módulos}
Na seguinte seção aprasentam-se desenhos comentados das estruturas de dados, 
todos os typedef e a documentação da API comentada, função a função.

\subsection{Catálogo Clientes}
\subsubsection{Definições de dados (Typedef)}
\emph{typedef struct node * ClientsCat;}
\subsubsection{API}
% initClients()
\noindent\textbf{ClientsCat initClients()} 
\par
Inicializa a estrutura do catálogo de clientes.  Para garantir o encapsulamento de dados a função devolve um valor do tipo \emph{ClientsCat} que o utilizador da API 
não sabe como está definido pois não sabe como está definida a \emph{struct node} \newline

% insertClient()
\noindent\textbf{ClientsCat insertClient (ClientsCat cat, char * client)}
\par Insere um dado client, o argumento \emph{client}, na estrutura de clientes, argumento \emph{cat}.
\par Se a estrutura não tiver sido inicializada ou o cliente não existir a função retornou o valor \emph{NULL},
caso contrário devolve o valor cat, isto permite que duas variáveis trabalhem na mesma estrutura.

\end{document}