\documentclass[12pt] {article}
\usepackage[portuguese]{babel}
\usepackage[utf8]{inputenc}
\setlength\parindent{24pt}

\begin{document}

\title{Relatório Trabalho Prático C \\ $\small{Grupo 13}$}
\author{João Costa \and Leandro Salgado \and Martinho Aragão}

\maketitle

%--------------------------------
% Group Members
%--------------------------------

\begin{flushleft}
João Costa A70563 \\
FOTO
\end{flushleft}

\begin{flushleft}
Leandro Salgado A70949 \\
FOTO
\end{flushleft}

\begin{flushleft}
Martinho Aragão AXXXX \\
FOTO
\end{flushleft}

\newpage

\tableofcontents

\newpage

\section{Módulos}
Na seguinte seção aprasentam-se desenhos comentados das estruturas de dados,
todos os typedef e a documentação da API comentada, função a função.

\subsection{Catálogo Clientes}
\par Esta subsecção trata da API do catálogo de clientes e da sua implementação
\subsubsection{Definições de dados (Typedef)}
\emph{typedef struct node * ClientsCat;} - Catálogo de Clientes

%-----------------------------
% Clients API
%-----------------------------
\subsubsection{Funções API}

% initClients
\noindent\textbf{ClientsCat initClients()}
\par
Inicializa a estrutura do catálogo de clientes.  Para garantir o encapsulamento de dados a função devolve um valor do tipo \emph{ClientsCat} que o utilizador da API
não sabe como está definido pois não sabe como está definida a \emph{struct node.} \\

% insertClient
\noindent\textbf{ClientsCat insertClient (ClientsCat cat, char * client)}
\par Insere um dado client, o argumento \emph{client}, na estrutura de clientes, argumento \emph{cat}.
\par Se a estrutura não tiver sido inicializada ou o cliente não existir a função retornou o valor \emph{NULL},
caso contrário devolve o valor cat, isto permite que duas variáveis trabalhem na mesma estrutura. \\

% searchClient
\noindent \textbf {Bool searchClient (char * client)}
\par Verifica se um cliente, o argumento \emph{client}, existe no catálogo de clientes.
\par A função devolve um valor do tipo Bool, definido em \emph{"Boolean.h"}, que será \emph{true} se o
cliente existir no catálogo e \emph{false} caso contrário. \\

% removeClient
\noindent \textbf{ClientsCat removeClient(ClientsCat cat, char * client)}
\par Remove um cliente, o argumento \emph{client}, de um dado catálogo de clientes, o argumento \emph{cat}.
\par A função retorna \emph{NULL} caso a estrutura não tenha sido inicializada ou o cliente seja inválido, e
retorna a própria estrutura caso contrário.

% searchClients
\noindent \textbf{StrList searchClients (ClientsCat cat, char init)}
\par Cria lista com todos os clientes cujo código comece por uma certa letra, o argumento \emph{init}, e que
estejam presentes no catálogo, \emph{cat}, passado como argumento.
\par Como o utilizador não conhece a definição do tipo \emph{StrList} apenas a main sabe como utilizar
o valor devolvido pela função. \\

% numOfClients
\noindent \textbf{int numOfClients (ClientsCat cat)}
\par Calcula o número de clientes presentes no catálogo, \emph{cat}, fornecido como argumento.
\par Como o utilizador não sabe como é que a estrutura de dados do catálogo está definida não consegue
calcular o número de clientes no catálogo sem recorrer a esta função. \\

% deleteCat
\noindent \textbf{ClientsCat deleteCat (ClientsCat cat)}
\par Liberta a memória ocupada pelo catálogo de clientes fornecido como argumento, \emph{cat}.
\par A função retorna NULL se a libertação de memória for bem sucedida. \\

% validateClient
\noindent \textbf{Bool validateClient (char * client)}
\par Verifica se um dado cliente, o argumento \emph{client}, é válido. Um cliente é válido se os dois primeiros
caracteres forem duas letras maiúsculas e os restantes três caracteres forem números.
\par A função retorn \emph{true} se o cliente for válido e \emph{false} se não o for.

\end{document}